\documentclass[aspectratio=169,12pt]{beamer}
\usetheme{CSCS}

\usepackage[no-math]{fontspec}
\usepackage{multicol}
\usepackage{xcolor}
\usepackage{xspace}
\usepackage{listings}

% Pick your font here
%\setsansfont{Fira Sans Condensed}
%\setsansfont{Catamaran}
\setsansfont{Helvetica}
\setmonofont{Courier}
%\setmonofont{Menlo}
\urlstyle{sf}

\graphicspath{{./figs/}}


% define footer text
\newcommand{\footlinetext}{Footer message}

% Select the image for the title page
\newcommand{\picturetitle}{cscs_images/cscs_entrance_painting.pdf}
%\newcommand{\picturetitle}{cscs_images/image5.pdf}
%\newcommand{\picturetitle}{cscs_images/image6.pdf}

% FIXME: I need to move the following style definitions inside the beamer style
% file

\lstdefinestyle{cxxstyle}{
  basicstyle=\scriptsize\ttfamily,
  keywordstyle=\color{blue},
  stringstyle=\color{magenta},
  language=[11]C++,
  showstringspaces=false,
  escapechar=\%
}

\lstdefinestyle{shstyle}{
  language=bash,
  basicstyle=\scriptsize\ttfamily,
  keywordstyle=\color{blue},
  stringstyle=\color{magenta},
  commentstyle=\itshape\color{cscsred},
}

\lstdefinestyle{pystyle}{
  language=Python,
  basicstyle=\ttfamily\tiny,
  keywordstyle=\color{blue},
  stringstyle=\color{magenta},
  commentstyle=\itshape\color{cscsred},
  %% numbers=left,
  %% stepnumber=5,
  %% numbersep=3pt,
  %% numberstyle=\footnotesize,
  morekeywords={as, self},
}


% Style for rendering ReFrame output
\lstdefinestyle{rfmstyle}{
  basicstyle=\tiny\ttfamily,
  keywordstyle=\color{green!60!black},
  keywordstyle=[2]{\color{red!80!black}},
  commentstyle=\itshape\color{cscsred},
  sensitive,
  keywords={RUN, OK, PASSED},
  keywords=[2]{FAIL, FAILED}
}

\newcommand\pyinline[1]{\lstinline[style=pystyle]!#1!}
\newcommand\shinline[1]{\lstinline[style=shstyle,basicstyle=\ttfamily\normalsize]!#1!}


% Please use the predifined colors:
% cscsred, cscsgrey, cscsgreen, cscsblue, cscsbrown, cscspurple, cscsyellow, cscsblack, cscswhite

\author{
  John Smith, Role, CSCS
}
\title{CSCS \LaTeX\ Presentation Template}
\subtitle{
  Event Name
}
\date{DD/MM/YYYY}

\begin{document}

% TITLE SLIDE
\cscstitle

\cscstableofcontents{TOC Title}

\section{Topic 1}
\subsection{Subtopic 1}
\subsection{Subtopic 2}
\subsubsection{Subsubtopic 1}
\subsubsection{Subsubtopic 2}

\section{Topic 2}
\subsection{Subtopic 1}
\subsubsection{Subsubtopic 1}


\begin{frame}{Slide title}{Slide subtitle}
  \begin{itemize}
  \item Point 1
    \begin{itemize}
    \item Subpoint 1
    \item Subpoint 2
    \end{itemize}
  \item Point 2
  \item Point 3
  \end{itemize}
\end{frame}

\part{Part Title}

% Block style example
\begin{frame}{Quick Styles}
\begin{columns}
    \begin{column}{0.12\paperwidth}
    \begin{black0block}{Abc}
    \end{black0block}
    \begin{black1block}{Abc}
    \end{black1block}
    \begin{black2block}{Abc}
    \end{black2block}
    \end{column}

    \begin{column}{0.12\paperwidth}
    \begin{green0block}{Abc}
    \end{green0block}
    \begin{green1block}{Abc}
    \end{green1block}
    \begin{green2block}{Abc}
    \end{green2block}
    \end{column}

    \begin{column}{0.12\paperwidth}
    \begin{blue0block}{Abc}
    \end{blue0block}
    \begin{blue1block}{Abc}
    \end{blue1block}
    \begin{blue2block}{Abc}
    \end{blue2block}
    \end{column}

    \begin{column}{0.12\paperwidth}
    \begin{brown0block}{Abc}
    \end{brown0block}
    \begin{brown1block}{Abc}
    \end{brown1block}
    \begin{brown2block}{Abc}
    \end{brown2block}
    \end{column}

    \begin{column}{0.12\paperwidth}
    \begin{purple0block}{Abc}
    \end{purple0block}
    \begin{purple1block}{Abc}
    \end{purple1block}
    \begin{purple2block}{Abc}
    \end{purple2block}
    \end{column}

    \begin{column}{0.12\paperwidth}
    \begin{yellow0block}{Abc}
    \end{yellow0block}
    \begin{yellow1block}{Abc}
    \end{yellow1block}
    \begin{yellow2block}{Abc}
    \end{yellow2block}
    \end{column}

    \begin{column}{0.12\paperwidth}
    \begin{red0block}{Abc}
    \end{red0block}
    \begin{red1block}{Abc}
    \end{red1block}
    \begin{red2block}{Abc}
    \end{red2block}
    \end{column}

\end{columns}
\end{frame}

% Example
\begin{frame}{Formula}{Subtitle}
    My formula is $E=m\times c^2$
    \begin{blue2block}{Are examples in a blue block?}
    \begin{itemize}
        \item My example
        \begin{itemize}
            \item for a formula
            \begin{itemize}
                \item rocks!
            \end{itemize}
        \end{itemize}
    \end{itemize}
    \end{blue2block}

    \begin{red2block}{Alert an error in the formula!}
    \begin{enumerate}
        \item First enumerated item
        \begin{enumerate}
            \item First first enumerated item
            \begin{enumerate}
                \item First first first enumerated item
            \end{enumerate}
        \end{enumerate}
        \item just joking \dots
    \end{enumerate}
    \end{red2block}
\end{frame}

\begin{frame}[fragile]{C/C++ code listings}{Example copied from \url{https://en.cppreference.com/w/cpp/algorithm/move}}
  \begin{lstlisting}[style=cxxstyle]
void f(int n)
{
    std::this_thread::sleep_for(std::chrono::seconds(n));
    std::cout << "thread " << n << " ended" << '\n';
}

int main()
{
    std::vector<std::thread> v;
    v.emplace_back(f, 1);
    v.emplace_back(f, 2);
    v.emplace_back(f, 3);
    std::list<std::thread> l;
    // copy() would not compile, because std::thread is noncopyable

    std::move(v.begin(), v.end(), std::back_inserter(l));
    for (auto& t : l) t.join();
}
  \end{lstlisting}
\end{frame}

\begin{frame}[fragile]{Python code listings}{Example copied from \url{https://github.com/eth-cscs/reframe/blob/master/tutorials/basics/hello/hello2.py}}
  \begin{lstlisting}[style=pystyle]
import reframe as rfm
import reframe.utility.sanity as sn


@rfm.parameterized_test(['c'], ['cpp'])
class HelloMultiLangTest(rfm.RegressionTest):
    def __init__(self, lang):
        self.valid_systems = ['*']
        self.valid_prog_environs = ['*']
        self.sourcepath = f'hello.{lang}'
        self.sanity_patterns = sn.assert_found(r'Hello, World\!', self.stdout)
  \end{lstlisting}
\end{frame}

\begin{frame}[fragile]{Shell code listings}{Example copied from \url{https://reframe-hpc.readthedocs.io/en/stable/tutorial_basics.html}}
  \begin{lstlisting}[style=shstyle]
#!/bin/bash
#SBATCH --job-name="rfm_StreamWithRefTest_job"
#SBATCH --ntasks=1
#SBATCH --output=rfm_StreamWithRefTest_job.out
#SBATCH --error=rfm_StreamWithRefTest_job.err
#SBATCH --time=0:10:0
#SBATCH -A csstaff
#SBATCH --constraint=gpu

module unload PrgEnv-cray
module load PrgEnv-gnu
export OMP_NUM_THREADS=4
export OMP_PLACES=cores
srun ./StreamWithRefTest
  \end{lstlisting}
\end{frame}

% THANK YOU SLIDE
\cscsthankyou{Thank you for your attention}

\end{document}
