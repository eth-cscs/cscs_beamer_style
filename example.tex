\documentclass[aspectratio=1610]{beamer}
\usetheme{CSCS}

% define footer text
\newcommand{\footlinetext}{Insert\_Footer}

% Select the image for the title page
\newcommand{\picturetitle}{cscs_images/image3.pdf}
%\newcommand{\picturetitle}{cscs_images/image5.pdf}
%\newcommand{\picturetitle}{cscs_images/image6.pdf}


% Please use the predifined colors:
% cscsred, cscsgrey, cscsgreen, cscsblue, cscsbrown, cscspurple, cscsyellow, cscsblack, cscswhite

\author{Author, CSCS}
\title{Multiline \\ presentation title}
%\title{Single line presentation title}
\subtitle{Event}
\date{\today}

\begin{document}

% TITLE SLIDE
\cscstitle

% EMPTY SLIDE
\begin{frame}
\end{frame}

% TABLE OF CONTENT SLIDE
% All options for table of contents:
% currentsection, currentsubsection, firstsection=xx, hideallsubsections, hideothersubsections, part=xx, pausesections, pausesubsections, sections=xx, sections={xx-yy}, sections={xx,yy}
%\cscstableofcontents[hideallsubsections]{Title}
\cscstableofcontents{TOC Title}

\section{Section1}
\subsection{Subsection1}
\subsubsection{Subsubsection1}

\section{Section2}
\subsection{Subsection2}
\subsubsection{Subsubsection2}

% New part; a frame with part name will be automatically inserted
\part{Part Title}

% Block style example
\begin{frame}{Quick Styles}
\begin{columns}
    \begin{column}{0.12\paperwidth}
    \begin{black0block}{Abc}
    \end{black0block}
    \begin{black1block}{Abc}
    \end{black1block}
    \begin{black2block}{Abc}
    \end{black2block}
    \end{column}

    \begin{column}{0.12\paperwidth}
    \begin{green0block}{Abc}
    \end{green0block}
    \begin{green1block}{Abc}
    \end{green1block}
    \begin{green2block}{Abc}
    \end{green2block}
    \end{column}

    \begin{column}{0.12\paperwidth}
    \begin{blue0block}{Abc}
    \end{blue0block}
    \begin{blue1block}{Abc}
    \end{blue1block}
    \begin{blue2block}{Abc}
    \end{blue2block}
    \end{column}

    \begin{column}{0.12\paperwidth}
    \begin{brown0block}{Abc}
    \end{brown0block}
    \begin{brown1block}{Abc}
    \end{brown1block}
    \begin{brown2block}{Abc}
    \end{brown2block}
    \end{column}

    \begin{column}{0.12\paperwidth}
    \begin{purple0block}{Abc}
    \end{purple0block}
    \begin{purple1block}{Abc}
    \end{purple1block}
    \begin{purple2block}{Abc}
    \end{purple2block}
    \end{column}

    \begin{column}{0.12\paperwidth}
    \begin{yellow0block}{Abc}
    \end{yellow0block}
    \begin{yellow1block}{Abc}
    \end{yellow1block}
    \begin{yellow2block}{Abc}
    \end{yellow2block}
    \end{column}

    \begin{column}{0.12\paperwidth}
    \begin{red0block}{Abc}
    \end{red0block}
    \begin{red1block}{Abc}
    \end{red1block}
    \begin{red2block}{Abc}
    \end{red2block}
    \end{column}

\end{columns}
\end{frame}

% Example
\begin{frame}{Formula}
    My formula is $E=m\times c^2$
    \begin{blue2block}{Are examples in a blue block?}
    \begin{itemize}
        \item My example
        \begin{itemize}
            \item for a formula
            \begin{itemize}
                \item rocks!
            \end{itemize}
        \end{itemize}
    \end{itemize}
    \end{blue2block}

    \begin{red2block}{Alert an error in the formula!}
    \begin{enumerate}
        \item First enumerated item
        \begin{enumerate}
            \item First first enumerated item
            \begin{enumerate}
                \item First first first enumerated item
            \end{enumerate}
        \end{enumerate}
        \item just joking \dots
    \end{enumerate}
    \end{red2block}
\end{frame}

\begin{frame}[fragile]{C/C++ Code listings}
\begin{Cpplisting}{My code title with some openmp}
int main (int argc, char *argv[]) {
    int nthreads, tid;
    #pragma omp parallel private(nthreads, tid)
    {
        tid = omp_get_thread_num();
        printf("Hello World from thread = %d\n", tid);
        if (tid == 0) {
            nthreads = omp_get_num_threads();
            printf("Number of threads = %d\n", nthreads);
        }
    }  /* All threads join master thread and disband */
    return 0;
}
\end{Cpplisting}
Here some C++ code within text: \lstinlineCpp{int i = foo();}
\end{frame}

\begin{frame}[fragile]{Fortran code listings}
\begin{Fortranlisting}{My Fortran code here}
program variable
real :: x ! This defines a variable called x
x = 23.6*log(4.2)/(3.0+2.1) ! This gives it a value
print *, 'The value of x is ', x 
end program
\end{Fortranlisting}
Here some Fortran code within text: \lstinlineFortran{INTEGER :: Y(1,5)}
\end{frame}

% THANK YOU SLIDE
\cscsthankyou{Thank you for your attention.}

\end{document}
